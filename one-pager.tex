% define document type, font-size, and paper dimensions
\documentclass[11pt, letterpaper]{article}
% set up image package & path
\usepackage{graphicx}
% subcaption for multi-figures
\usepackage{subcaption}
% adds a header onto every page
\usepackage{fancyhdr}
% adds indent to paragraph after section header
\usepackage{indentfirst}
% make margins 1in
\usepackage[total={6.5in,8.75in},
top=1.25in, left=1in]{geometry}
% add hyperlinks
\usepackage{hyperref}

% set up page style & graphic path
\pagestyle{fancy}
\graphicspath{{./images/}}

% Resize table rows
\renewcommand{\arraystretch}{1.75}

% Set up title page
\renewcommand{\maketitle}{
    \begin{titlepage}
        % Set up title page
        \centering
        \huge{\textbf{ECS 171 Group 10 Project}} \par
        \vspace{.5cm}
        % group leader
        \large{Leader: Calvin Chen} \par
        \vspace{.5cm}
        % group members
        \normalfont{Amira Basyouni, Alexis Lydon, Calvin Chen, Ryan Yu, Tianming Tan} \par
        \vspace{.5cm}
        % date
        \today
    \end{titlepage}
}

% set up header
\fancyhf{} % clear all header and footer fields
\lhead{ECS 171 Group Project} % left header
\rhead{\thepage} % right header
%\fancyhfoffset[LR]{1cm} % increase left, right margin by 1cm

\begin{document}
    \maketitle
    
    \newpage

    \section*{Problem Statement}
    As a student, we are in a battle between the many stressors we have in our lives 
    and our strive towards academic success. We are faced with deadlines, exams, and 
    the pressure to perform well. These factors can have a negative impact on our 
    quality of learning and thus hinder our progress towards acquiring a degree and 
    entering the workforce. Our team would like to alleviate this problem using sleep. 
    We would like to predict the effects poor sleep quality can have on our stress 
    levels. Would longer sleep duration result in a decrease in stress? After we 
    analyze our data, we will be able to confirm the correlation between those two 
    attributes. We will also look into other factors affecting stress such as physical 
    activity, occupation, and blood pressure. By finding various correlations, we can 
    narrow down the factors affecting stress and ultimately propose a healthy set of 
    habits to help manage our mental health. The data will involve ages that vary from 
    27 to 59 years of age. Although it does not cover the typical age of a college 
    student, our analysis can still help us build awareness that would last us for 
    years to come.

    \section*{Dataset}
    The dataset being used in this project is the "Sleep Health and Lifestyle Dataset" 
    by Laksika Tharmalingam on Kaggle. The dataset comes in the form of a CSV file and 
    contains 400 rows and 13 columns, encompassing various sleep and lifestyle variables. 
    These include gender, age, occupation, sleep duration, sleep quality, physical 
    activity, stress levels, BMI category, blood pressure, heart rate, daily steps, 
    and sleep disorder status. We chose this dataset because, as students, we can 
    relate to the common sleep issues many of us face. Our project focuses on 
    investigating the relationship between sleep and health, making a dataset on 
    sleep and health the most suitable choice. A limitation of the dataset is its 
    synthetic nature, generated artificially rather than from real-world observations. 
    While it may lack some real-world nuances, high-quality synthetic data can 
    effectively train and test machine learning models in this context. Here is the link
     of our dataset for further reference: \href{https://www.kaggle.com/datasets/uom190346a/sleep-health-and-lifestyle-dataset/data}{Sleep Health and Lifestyle Dataset}
    
    \section*{Goals}
    In this machine learning project centered on the intersection of sleep health and 
    stress, we aim to achieve several key objectives to better understand and predict 
    the dynamics between these two factors. First, we will dive into the relationship 
    between sleep duration and stress. Our investigation will involve analyzing how 
    varying durations of sleep impacts an individual's stress levels. Next, we will 
    explore the influence of sleep quality on stress levels. As an integral part of the 
    project, we will develop an accurate regression model for stress prediction based 
    on factors of sleep quality. We will also develop models that predict stress levels 
    based on other highly correlated attributes in our dataset other than just sleep.
    Our last goal is somewhat of a reach goal, and it involves investigating the amount 
    of extra sleep required for stress reduction. These defined project goals will be a 
    structured guide throughout our research, data collection, analysis, and model 
    development. They will enable us to gain valuable insights into the complex 
    relationship between sleep health and stress levels while delivering solutions and 
    predictions for this area of health.

    \newpage

    \section*{Project Timeline}
    For our project, we plan to roughly follow the timeline below:
    
    \begin{table}[htbp]
        \centering
        \begin{tabular}{l|c|c|c}
            \hline \hline
            \multicolumn{4}{c}{\textbf{Project Timeline}} \\ % resizes n columns into one
            \hline \hline
            \multicolumn{1}{c|}{\textbf{Task}} & \textbf{Start Date} & \textbf{End Date} & \textbf{Duration (days)} \\
            \hline
            Background/Literature Review & Oct. 16 & Oct. 22 & 7 \\
            \hline
            Exploratory Data Analysis & Oct. 23 & Oct. 29 & 7 \\
            \hline
            Developing prediction models & Oct. 30 & Nov. 5-8 & 7-10 \\
            \hline
            Evaluation of the model and testing performance & Nov. 6-9 & Nov. 12-14 & 7-9 \\
            \hline
            Developing front-end to display and run models & Nov. 13-15 & Nov. 19-21 & 7 \\
            \hline
            Finish Report and Minor Adjustments & Nov. 20-22 & Nov. 23-25 & 4 \\
            \hline
            Presentation Practice & Nov. 24-26 & Nov. 28-30 & 5 \\
            \hline \hline
        \end{tabular}
        \caption{Tasks, Start and End Dates, and Planned Duration}
        \label{tab:timeline}
    \end{table}

    \textbf{\underline{Task Breakdown}}
    \begin{enumerate}
        \item Background/Literature Review: Research our problem and describe it scientifically
        \item Exploratory Data Analysis: Understanding our dataset and filtering out noise
        \item Developing prediction models: Creating 3+ models to quantify and predict desired outcomes
        \item Evaluation of the model and testing performance: Testing our models and deciding on the best performing one
        \item Developing front-end to display and run models: Invoke our models and display results
        \item Finish Report and Minor Adjustments: Clean up our code and final report paper
        \item Presentation Practice: Practice our presentation and prepare for questions
    \end{enumerate}

\end{document}
