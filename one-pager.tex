% define document type, font-size, and paper dimensions
\documentclass[11pt, letterpaper]{article}
% set up image package & path
\usepackage{graphicx}
% subcaption for multi-figures
\usepackage{subcaption}
% adds a header onto every page
\usepackage{fancyhdr}
% adds indent to paragraph after section header
\usepackage{indentfirst}
% make margins 1in
\usepackage[total={6.5in,8.75in},
top=1.25in, left=1in]{geometry}
% add hyperlinks
\usepackage{hyperref}

% set up page style & graphic path
\pagestyle{fancy}
\graphicspath{{./images/}}

% Resize table rows
\renewcommand{\arraystretch}{1.75}

% Set up title page
\renewcommand{\maketitle}{
    \begin{titlepage}
        % Set up title page
        \centering
        \huge{\textbf{ECS 171 Group 10 Project}} \par
        \vspace{.5cm}
        % group leader
        \large{Leader: Calvin Chen} \par
        \vspace{.5cm}
        % group members
        \normalfont{Amira Basyouni, Alexis Lydon, Calvin Chen, Ryan Yu, Tianming Tan} \par
        \vspace{.5cm}
        % date
        \today
    \end{titlepage}
}

% set up header
\fancyhf{} % clear all header and footer fields
\lhead{ECS 171 Group Project} % left header
\rhead{\thepage} % right header
%\fancyhfoffset[LR]{1cm} % increase left, right margin by 1cm

\begin{document}
    \maketitle
    
    \newpage

    \section*{Problem Statement}
    Sleep, as a student, many of us don't get enough of it. We've heard it 
    affects our health, but in what ways? How much is enough sleep and why 
    should we care? After analyzing our data, we can hopefully understand 
    why this is important and improve our sleep quality. We will be able to 
    narrow down the factors that affect our sleep, and build habits that can 
    mitigate and reduce the negative effects sleep deprivation has on our 
    health. Through the use of Machine Learning models, we can create 
    predictions on what factors would lead to better or worse sleep, 
    and predict if a person would have high or low sleep quality based 
    on these factors. Not only will this help students like us 
    better manage our sleep, but it will also help us as a community of 
    diverse occupations to get a good night's rest.

    \section*{Dataset}
    The dataset being used in this project is the "Sleep Health and Lifestyle Dataset" by Laksika Tharmalingam on Kaggle. 
    The dataset comes in the form of a CSV file and contains 400 rows and 13 columns, encompassing various sleep and lifestyle variables. 
    These include gender, age, occupation, sleep duration, sleep quality, physical activity, stress levels, BMI category, blood pressure, heart rate, daily steps, 
    and sleep disorder status. We chose this dataset because, as students, we can relate to the common sleep issues many of us face. Our project focuses on investigating 
    the relationship between sleep and health, making a dataset on sleep and health the most suitable choice. A limitation of the dataset is its synthetic nature, generated 
    artificially rather than from real-world observations. While it may lack some real-world nuances, high-quality synthetic data can effectively train and test machine learning 
    models in this context. Here is the link of our dataset for further reference: \href{https://www.kaggle.com/datasets/uom190346a/sleep-health-and-lifestyle-dataset/data}{Sleep Health and Lifestyle Dataset}
    
    \section*{Goals}
    In this machine learning project centered on the intersection of sleep health and stress, we aim to achieve several key objectives to better understand and predict the dynamics 
    between these two factors. First, we will dive into the relationship between sleep duration and stress. Our investigation will involve analyzing how varying durations of sleep 
    impacts an individual's stress levels. Next, we will explore the influence of sleep quality on stress levels. As an integral part of the project, we will develop an accurate 
    regression model for stress prediction based on factors of sleep quality. Another aspect of our project involves investigating the minimum adequate sleep duration required 
    for stress reduction. Lastly, we will also examine the relationship between various sleep patterns including duration and quality, and an individual's blood pressure. These 
    defined project goals as a structured guide throughout our research, data collection, analysis, and model development. They will enable us to gain valuable insights into the 
    complex relationship between sleep health and stress levels while delivering solutions and predictions for this area of health.

    \newpage

    \section*{Project Timeline}
    For our project, we plan to roughly follow the timeline below:
    
    \begin{table}[htbp]
        \centering
        \begin{tabular}{l|c|c|c}
            \hline \hline
            \multicolumn{4}{c}{\textbf{Project Timeline}} \\ % resizes n columns into one
            \hline \hline
            \multicolumn{1}{c|}{\textbf{Task}} & \textbf{Start Date} & \textbf{End Date} & \textbf{Duration (days)} \\
            \hline
            Background/Literature Review & Oct. 16 & Oct. 22 & 7 \\
            \hline
            Exploratory Data Analysis & Oct. 23 & Oct. 29 & 7 \\
            \hline
            Developing prediction models & Oct. 30 & Nov. 5-8 & 7-10 \\
            \hline
            Evaluation of the model and testing performance & Nov. 6-9 & Nov. 12-14 & 7-9 \\
            \hline
            Developing front-end to display and run models & Nov. 13-15 & Nov. 19-21 & 7 \\
            \hline
            Finish Report and Minor Adjustments & Nov. 20-22 & Nov. 23-25 & 4 \\
            \hline
            Presentation Practice & Nov. 24-26 & Nov. 28-30 & 5 \\
            \hline \hline
        \end{tabular}
        \caption{Tasks, Start and End Dates, and Planned Duration}
        \label{tab:timeline}
    \end{table}

    \textbf{\underline{Task Breakdown}}
    \begin{enumerate}
        \item Background/Literature Review: Research our problem and describe it scientifically
        \item Exploratory Data Analysis: Understanding our dataset and filtering out noise
        \item Developing prediction models: Creating 3+ models to quantify and predict desired outcomes
        \item Evaluation of the model and testing performance: Testing our models and deciding on the best performing one
        \item Developing front-end to display and run models: Invoke our models and display results
        \item Finish Report and Minor Adjustments: Clean up our code and final report paper
        \item Presentation Practice: Practice our presentation and prepare for questions
    \end{enumerate}

\end{document}
